\setchapterimage[2cm]{../images/header-methods.jpg}
\setchapterpreamble[u]{\margintoc}
\chapter{Computational methods and specifications}
\labch{methods}


\section{General techniques}



\begin{table*}[h]
    \centering
    \caption{Computational techniques}
    \label{tab:computational_techinques}
    \begin{tabular}{@{}lllll@{}}
        \toprule
        Calculation & Technique & Spec. & Functional & Basis set \\
        \midrule
        Geometry optimizations & DFT & & M06-2X & def2TZVPP \\
        NICS calculations & DFT & GIAO & b3lyp & \fix{base} \\
        \bottomrule
    \end{tabular}
\end{table*}
\blindtext


\section{Software}

\blindtext


\section{Hardware}
All of the calculations were perfomed using either the Centro de Supercomputación de Galicia's (CESGA) infraestructures, or the propietary cluster of the S3 research group.

S3's cluster is composed of nodes that include 16 Intel\textregistered Xeon\textregistered CPU E5-2630 v3 2.40GHz cores and 60 GB of RAM.

CESGA's
