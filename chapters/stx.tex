\setchapterimage[2cm]{../images/header-stx.jpg}
%\setchapterpreamble[u]{\margintoc}
\chapter{Sunflower-saxitoxin complexes}
\labch{stx}

Having obtained a general characterization of the sunflower-type molecules, it's time to get back to the problem at hand and start looking into how they can be applied.

Let's reintroduce the molecule that motivated this whole study: saxitoxin (STX).
STX is

\section{Spectroscopic study of lone STX}

\section{Study of adsorption}
\blindtext
\subsection{Sampling and optimization}
\blindtext
\subsection{Maxwell-Boltzmann statistics}
\blindtext
\subsection{Basis Set Superposition Error correction}
\blindtext

\section{Study of non-covalent interactions}
\blindtext

\section{Study of UV behavior}
\blindtext
\subsection{General UV spectroscopy}
\blindtext
\subsection{Charge transfer analysis}
\blindtext

\section{Resonance Raman}
\blindtext
\subsection{Generation and comparison of spectra}
\blindtext
\subsection{Combined resonance graphs}
\blindtext
\subsection{Final selection}
\blindtext
