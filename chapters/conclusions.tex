\setchapterimage[2cm]{../images/header-conclusions.png}
%\setchapterpreamble[u]{\margintoc}
\chapter{Summary and final conclusions}
\labch{conclusions}

All in all, the development and results of the work were considered satisfactory.
Translating the proposals made throughout this work to the real world would be no easy task as it would involve many steps and many different areas of expertise: the design of effective synthetic routes for the sunflowers, the preparation of portable devices that could carry them, the selection of a technique to obtain samples contaminated by STX, the development and tuning of analytical techniques based on the resonance effect...
In any case, all of these challenges could be pointed in the right direction and made easier thanks to this previous computational exploration.

Considering only theory and computation, however, the techniques already discussed in this work could be expanded, improved and continued in many ways, such as the following:

\begin{itemize}
    \item Studying the possibilities of using different methods or larger basis sets. The selected calculation level was considered to be a good fit for the problem at hand considering the number of calculations that needed to be done and the available computational power. However, now that the problem is more mature and well known, some of its parts could be better described by translating them to higher levels of computation
    \item Complementing the description of the aromaticity of the sunflowers using alternative techniques such as the study of the anisotropy of the current induced density\sidenote{Most commonly known as ACID\cite{geuenich17,geuenich17-2}}. Aromaticity is a complex property with many dimensions, and observing it from additional points of view could only improve its understanding
    \item Expanding and packaging the NICS Python program so it can easily be distributed and applied to study the NICS profile of any non-planar surface-like part of a molecule
    \item Delving deeper into the nature and composition of the electronic transitions of the flowers, the STX and the flower-STX complexes
    \item In a similar line of research, analyzing the transfers of charge that occur within the complexes when these electronic excitations happen. This could serve as a way to explain the magnitude and nature of the observed resonance effects
    \item Investigating the possible fluorescence effects that could arise in a real life resonance Raman measurement using certain laser wavelengths
    \item Determining the nature of the possible implications of SERS effects in the Raman amplifications
    \item Expanding the study to flowers with other different heteroatoms and/or other numbers of petals
    \item Determining the nature of the flower-STX system's non-covalent interactions
    \item Studying the adsorption of the STX to the flowers in a more realistic setting by adding the simulation of a solvent to the calculations
    \item Simulating the interactions that would occur in a system with many molecules of sunflower and STX
\end{itemize}

As many of these suggestions were well beyond the scope or computational availability of this work, they remain as open lines of research, ready to be expanded upon in the future.
At the moment, however, the level of depth attained in this text in particular was considered sufficient for its objectives as an end of master's theses.

Finally, as a way to close off this little book, a piece of the author's mind on the general purpose of a work like this:
The aim of \refch{sunflowers} was to provide insight into this novel family of sunflowers, but also to illustrate a series of characterization techniques that could be utilized to describe a variety of different surface-like systems.
In a similar way, while STX was the protagonist of \refch{stx} due to the previous studies in which the research group was involved, the flowers could be applied to a number of different toxins or molecules with analytical interest.
The procedures shown and proposed all throughout this work are meant to serve as suggestions and references to similar studies: keeping their purposes, strengths and limitations in mind, they could be made useful in many different contexts; and even serve as starting points in the development of better and more sophisticated methodologies well beyond the scope of the author's reach.
