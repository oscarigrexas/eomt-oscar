\setchapterstyle{kao}
\setchapterpreamble[u]{\margintoc}
\chapter{Mathematics and Boxes}
\labch{mathematics}

\section{Theorems}

Despite most people complain at the sight of a book full of equations, 
mathematics is an important part of many books. Here, we shall 
illustrate some of the possibilities. We believe that theorems, 
definitions, remarks and examples should be emphasised with a shaded 
background; however, the colour should not be to heavy on the eyes, so 
we have chosen a sort of light yellow.\sidenote{The boxes are all of the 
same colour here, because we did not want our document to look like 
\href{https://en.wikipedia.org/wiki/Harlequin}{Harlequin}.}

\begin{definition}
\labdef{openset}
Let $(X, d)$ be a metric space. A subset $U \subset X$ is an open set 
if, for any $x \in U$ there exists $r > 0$ such that $B(x, r) \subset 
U$. We call the topology associated to d the set $\tau\textsubscript{d}$ 
of all the open subsets of $(X, d).$
\end{definition}

\refdef{openset} is very important. I am not joking, but I have inserted 
this phrase only to show how to reference definitions. The following 
statement is repeated over and over in different environments.

\begin{theorem}
A finite intersection of open sets of (X, d) is an open set of (X, d), 
i.e $\tau\textsubscript{d}$ is closed under finite intersections. Any 
union of open sets of (X, d) is an open set of (X, d).
\end{theorem}

\begin{proposition}
A finite intersection of open sets of (X, d) is an open set of (X, d), 
i.e $\tau\textsubscript{d}$ is closed under finite intersections. Any 
union of open sets of (X, d) is an open set of (X, d).\marginnote{You can even insert footnotes inside the theorem 
	environments; they will be displayed at the bottom of the box.}
\end{proposition}

\begin{lemma}
A finite intersection\footnote{I'm a footnote} of open sets of (X, d) is 
an open set of (X, d), i.e $\tau\textsubscript{d}$ is closed under 
finite intersections. Any union of open sets of (X, d) is an open set of 
(X, d).
\end{lemma}

You can safely ignore the content of the theorems\ldots I assume that if 
you are interested in having theorems in your book, you already know 
something about the classical way to add them. These example should just 
showcase all the things you can do within this class.

